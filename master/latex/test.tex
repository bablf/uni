\documentclass[11pt,a4paper,twoside,openright]{scrbook}
\usepackage{clba}

% Per Kapitel Nummerierung von Graphiken und Tabellen
\usepackage{chngcntr}
\counterwithin{figure}{chapter}
\counterwithin{table}{chapter}


% Hier die eigenen Daten eintragen
\global\fach{Computerlinguistik}
\global\arbeit{Masterarbeit}
\global\titel{Titel der Arbeit}
\global\bearbeiter{Florian Babl}
\global\betreuer{Dr. Philipp Dufter}
\global\pruefer{Prof. Dr. Hinrich Schütze}
\global\universitaet{Ludwig- Maximilians- Universität München}
\global\fakultaet{Fakultät für Sprach- und Literaturwissenschaften}
\global\department{Department 2}

\global\abgabetermin{26. Juli 2021}
\global\bearbeitungszeit{8. März - 26. Juli 2021}
\global\ort{München}


\begin{document}

% Deckblatt
\deckblatt

\pagestyle{scrheadings}
\pagenumbering{gobble}

% Erklärung fürs Prüfungsamt
\erklaerung

% Zusammenfassung
\addchap{Abstract}
\thispagestyle{scrplain}
\noindent

(hier max. 250 Wörter)

% Inhaltsverzeichnis
\pagenumbering{Roman}

\tableofcontents

% Text mit arabischer Nummerierung
\pagenumbering{arabic}

\chapter{Kapitel Eins}

\section{Ein Abschnitt}
Mein Name ist Hase und ich weiß von nichts. Das ist ein Testtext. Mein Name ist
Igel und ich weiß auch von nichts.

\subsection{Ein Unterabschnitt}
Blabla. Hier ein Unterabschnitt.

\subsubsection{Ein Unterunterabschnitt}
\label{sec:a}
Blabla. Hier ein Unterunterabschnitt.

\subsubsection{Noch ein Unterunterabschnitt}
\label{sec:b}

Wer \ref{sec:a} sagt, muss auch \ref{sec:b} sagen.

\subsection{Noch ein Unterabschnitt}

Das ist ein gewöhnlicher Absatz.

\paragraph{Ein Absatz mit Titel}
Paragraphen gibts auch.

\subparagraph{Ein Unterabsatz mit Titel}
Und dann auch noch Unterparagraphen.

\subsection*{Ein nicht nummerierter Unterabschnitt}
Dieser Unterabschnitt erscheint nicht im Inhaltsverzeichnis.
\newpage

\section{Beispiele}
Blabla.
\newpage

\chapter{Introduction to Chess}
\section{Mehr Beispiele}
\chapter{What is Language Modeling}
State of the art LM.
\section{Intorduciton to GPT-2}
why gpt 2
\chapter{Chess as a Setting}
PGN UCI Notation- Most important Chess terms
Why this setting. Problems and Challenges
\chapter{Introduction to Models}
# Todo: Auf welche anderen Modelle habe ich zugriff?
Pretrained Models:
Nur Blindfolded Paper -

Finetuned GPT2 Models:
1. GPT auf PGN Daten finetunen (10\% mistakes according to chess transformer paper aka Noever)
2. GPT auf UCI Daten finetuned (selfbuild)
2. kann GPT auf konvertierten PGN daten trainieren (um Feldman Paper nachzuahmen)

Comparison
\chapter{Datapreprocessing}
Kingbase und milibrary combined. Kingbase converted with awk command:
\begin{verbatim}
`awk '{split($0,a,"] ");print a[3]"]";print a[2]"]";print a[1]"]";print "";print a[4]; print "";}'
\end{verbatim}


\chapter{Introduction to Probing Methods}
problems with probing classifiers? Is this a legitimate way of checking if the model has learnded anything about the real world.

Learning probing with extracted hidden state
Few/one/zero shot training.
\chapter{Introduction to Probing Classifiers}
\chapter{Introduction to the task}
What I want to present? Generall research question. Not "how do transformers learn".
more like "do transformers learn this?" Is this information in the hidden state


\chapter{Results of Probing Classifiers}
\chapter{Comparison of Model Results}
\chapter{Consequence of Results}


commands for splitting dataset into train, val test
wc -l probing_dataset.jl
500000 probing_dataset.jl
flo@flo-PC:~/uni/master/code/data$ bc <<< "500000 /10*6"
300000
flo@flo-PC:~/uni/master/code/data$ bc <<< "500000 /10*2"
100000
flo@flo-PC:~/uni/master/code/data$ head -n 300000 probing_dataset.jl > train_probing.jl
flo@flo-PC:~/uni/master/code/data$ sed -n 300001,400000 probing_dataset.jl > val_probing.jl
sed: -e expression #1, char 13: missing command
flo@flo-PC:~/uni/master/code/data$ sed -n 300001,400000p probing_dataset.jl > val_probing.jl
flo@flo-PC:~/uni/master/code/data$ sed -n 400001,500000p probing_dataset.jl > val_probing.jl
flo@flo-PC:~/uni/master/code/data$ sed -n 300001,400000p probing_dataset.jl > val_probing.jl
flo@flo-PC:~/uni/master/code/data$ sed -n 400001,500000p probing_dataset.jl > test_probing.j


%Beispielliteratur
\begin{thebibliography}{9}
\bibitem{blindfolded}
\bibitem{Erdos01} P. Erd\H os, \emph{A selection of problems and
results in combinatorics}, Recent trends in combinatorics (Matrahaza,
1995), Cambridge Univ. Press, Cambridge, 2001, pp. 1--6.

\bibitem{ConcreteMath}
R.L. Graham, D.E. Knuth, and O. Patashnik, \emph{Concrete
mathematics}, Addison-Wesley, Reading, MA, 1989.

\bibitem{Knuth92} D.E. Knuth, \emph{Two notes on notation}, Amer.
Math. Monthly \textbf{99} (1992), 403--422.

\bibitem{Simpson} H. Simpson, \emph{Proof of the Riemann
Hypothesis},  preprint (2003), available at
\url{http://www.math.drofnats.edu/riemann.ps}.
\end{thebibliography}
\newpage

% Abbildungsverzeichnis (kann auch nach dem Inhaltsverzeichnis kommen)
\listoffigures
\newpage

% Tabellenverzeichnis (kann auch nach dem Inhaltsverzeichnis kommen)
\listoftables
\newpage

\addchap{Inhalt der beigelegten CD}
Code
\end{document}
